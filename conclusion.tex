En el presente trabajo se propuso un estudio de las técnicas y metodologías actuales de análisis inteligente de datos y visualización para la asistencia en la investigación criminal. Todo ello a partir de los registros de actividades delictivas, sus autores y las relaciones de datos que pueden derivarse a partir de ellas. Fue de especial interés la identificación de redes ilegales, tales como bandas delictivas o criminales para propender a una persecución penal inteligente.

Del estudio propuesto, como se explicó en los capítulos anteriores, se desarrolló un módulo de software como herramienta gráfica para visualizar la red de grupos de pertenencia de los actores delictuales.
El desarrollo de software se implementó en el mismo Ministerio Público Fiscal, del cual se tomaron los datos para generar los datasets de pruebas. De esta manera se ha logrado no sólo estudiar las técnicas y metodologías expuestas, sino también alcanzar la puesta en producción y uso de la herramienta, por los propios actores de la investigación.

Las primeras impresiones de aquellos especialistas de investigaciones penales han sido muy satisfactorias y permiten evaluar a este trabajo como el inicio de futuros desarrollos visuales para el apoyo a la toma de decisiones en la investigación penal.
