%MEJORAS PROPIAS AL ALGORITMO DE PAGERANK
\subsubsection{Mejoras a PageRank - Peso a los enlaces}

La idea detrás de PageRank es que las "buenas" páginas hacen referencia a otras "buenas" páginas. Por lo tanto, las páginas a las que hacen referencia esas "buenas" páginas tienen un PageRank más alto. Suponiendo que un usuario navega por la Web de forma aleatoria, de modo que, si está en una página, con cierta probabilidad se aburre y abandona la página, o elige de manera uniforme y aleatoria seguir uno de los enlaces de la misma página en la que se encuentra (eliminando los autoenlaces). Por lo tanto, la probabilidad de estar en la página "p" es

\begin{equation} 
	\label{eqn:ecuacionWLRank1} 
	PR(p) = \frac{q}{T} + (1 - q) \sum_{i} \frac{PR(r_i)}{L(r_i)} 
\end{equation}

donde T es el número total de páginas, q es la probabilidad de salir de la página p (en el trabajo original de PageRank se sugiere q = 0:15), ri son las páginas que apuntan a la página p, y L(ri) es el número de enlaces en la página ri. Estos valores se pueden usar como clasificación de páginas y se pueden calcular mediante un algoritmo iterativo que converge bastante rápido, ya que estamos interesados en el orden de clasificación en lugar de los valores de clasificación reales. El término q se denomina factor de amortiguamiento, ya que disminuye exponencialmente el spam de enlaces basado en secuencias de enlaces que regresan a una página.

De aquí surge una variante al algoritmo de PageRank original de Google, llamada WLRank propuesta en el trabajo de Ri Baeza-Yates y Emilio Davies \cite{baeza2004web}.

WLRank (Weighted Links Rank) asigna el valor de clasificación R(i) a la página i usando las siguientes ecuaciones:

\begin{equation} 
	\label{eqn:ecuacionWLRank2} 
	R(i) = \frac{q}{T} + (1 - q) \sum_{j} \frac{W(j,i)R(j)}{\sum_{k}W(j,k)} 
\end{equation}

\begin{equation} 
	\label{eqn:ecuacionWLRank3} 
	W(j,i) = L(j,i)(c+T(j,i)+AL(j,i)+RP(j,i))
\end{equation}

donde dado un enlace de la página j a la página i se tiene:

L(j; i) es 1 si el enlace existe, o 0 en caso contrario, y c es una constante que da un peso base a cada enlace,
T(j; i) es un valor que depende de la etiqueta donde se inserta el enlace,
AL(j;i) es la longitud del texto "ancla" del enlace dividida por una constante d que depende que estima la longitud promedio del texto ancla en caracteres, y RP(j;i) es la posición relativa del enlace en la página ponderado por una constante b.

Al igual que en PageRank, R(i) corresponde a la probabilidad de llegar a la página i mientras navega por la Web. Si W(j; i) = L(j; i) tenemos el PageRank original. Los cambios se explican a continuación. El término T(j; i) es una secuencia de constantes dependiendo de la etiqueta donde se encuentre el enlace. Por ejemplo, si el enlace está dentro de una etiqueta <h1>, tendrá un valor alto de T(j; i), un poco menos para <h2>, etc. Lo mismo para otras etiquetas de énfasis como <strong> o <b> .

El término AL(j;i) da más valor a los enlaces en los que el creador explica con más detalle a qué recurso Web se está enlazando. Por ejemplo, esto le da menos peso a los enlaces descritos con home o aquí. Finalmente, el término RP(j; i) da más peso a los enlaces que están al principio de la página que al final de la página (físicamente en el código HTML, no necesariamente en la vista del navegador).

Gracias a esta mejora sobre la fórmula original de PageRank es posible entonces darle mayor consideración en la fórmula a aquellos enlaces que tienen más pesos que otros.

Se procedió a adecuar nuestra fórmula de modo de contemplar los pesos en los enlaces y de esta forma realizar ciertas pruebas simples que comprueben que la modificación realizada consigue ajustar a nuestro modelo en una mejora respecto al algoritmo de PageRank original.

A continuación se muestra el ejemplo de 6 nodos mostrado con anterioridad en donde se puede observar el resultado ponderando los enlaces.

\begin{figure}
	\centering
	\includegraphics[width=0.50\linewidth]{6nodos-pageRank-peso-enlaces.png}
	\caption{Resultado de ejecución de PageRank para 6 nodos con modificación WLRank.} 
	\label{fig:6nodos-pageRank-peso-enlaces}
\end{figure}

Se puede apreciar a simple vista que ahora al tener peso las relaciones en la fórmula el resultado es mucho más preciso, existen muchos casos en cómun entre 145262 y 116587, lo que justifica la modificación en el ranking general. El nodo 
145262 tiene 1 caso en cómún con 132310 y 18 con 116587, un total de 19 relaciones a ese nodo. Por otra parte el nodo que quedó en segunda posicición en el ranking (145053), tiene un total de 15 enlaces provenientes de 13 casos en común con el nodo 129137 y de 1 caso en cómun con el nodo 116587 y con el nodo 123109.

\subsubsection{Mejoras a PageRank - Peso a los nodos}

